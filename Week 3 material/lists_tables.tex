\documentclass[12pt]{article}
\usepackage{amsmath}
\usepackage{graphicx}
\title{GEOPUG: Week 3\\Lists and Tables}
%\author{Jacob Richardson}
\usepackage{url}
\date{}

\begin{document}
\maketitle

\section{Lists}

The two types of lists we commonly use are bulleted and numbered lists.

\subsection{Bulleted List}

\begin{itemize} %Enter the bulleted list environment
\item %add a new marker
All lists use the \textbackslash item command to create a new marker. For Instance, \textbackslash item is used after this sentence to create a new bullet point.

\item Bulleted lists are created with the \textit{itemize} environment. I think of a grocery list as an itemized list. It has items but each item in my spaghetti recipe is as important as all the others.

\item Again, \textbackslash item creates a new marker in all lists. If I were to try to start a new paragraph by leaving a blank line in my write up\dots

I will see that lists in \LaTeX can easily handle multiple paragraphs per marker!
\end{itemize}

\subsection{Numbered List}

\begin{enumerate} %Begin the Numbered list environment
	\item Numbered lists are created within the \textit{enumerate} environment. 
	\item Each new marker will, by default be the next arabic numeral.
	\item We can also nest lists! All that is needed is to begin a new list environment of our choosing within the current list environment.
	
	\begin{enumerate} %Begin a Nested numbered list
		\item As you can see, nested enumerated lists use lower case letters by default.
		\item Nested itemized lists change to hyphens instead of bullets.
		\item An enumerated list nested inside an itemized list will use numbers. It is definitely something to play around with
	\end{enumerate}
	
	%white space can be left here and elsewhere in lists, which is different than the relatively picky equation environment.
\end{enumerate}

\section{Special Characters}
Special Characters are written out in a logical way, perhaps like lego bricks. Special character sets in MS Word-like documents have created specific characters before--hand, which don't always have what you want. \LaTeX gets around this by creating all special characters on-the-fly when you design them.

T\={o}hoku

Volc\'an Po\'as

``Hello World!'': You cannot paste pre-formatted quotation marks from MS Word like documents. 

,,Wilkommen zum \LaTeX, Fraulein Gr\"oning''

```Hello World,' Aur\'elie said''

\textit{This is all italic font.}

\textbf{This is all bold, \textit{and this is bold and italic font}}

hy-phen: e.g. 1-2.5

en--dash: e.g. I work half--time.

em---dash: e.g. She always thought---against all logic, but nevertheless---that she should learn \LaTeX.

Backslashes, underscores, dollar signs, ampersands, and percent signs are all special characters in \LaTeX. To create a backslash, we have to type \textbackslash textbackslash: \textbackslash. To use the rest, we have to use an additional backslash in front of them: the online folder puppy\_pictures costs \$3.00 to access \& 50\% of your time, when you should be working.

Urls can be created if we call the \textit{url} package in the preamble. \url{http://nvoss.myweb.usf.edu/puppy_pictures} %requires package "url"

\newpage %creates a new page
\thispagestyle{empty} %removes the page number on the current page only

\section{Tables}
Notice this starts on a new page and the page doesn't have a number. Done by using \textit{\textbackslash newpage} and \textit{\textbackslash thispagestyle\{empty\}}.

\subsection{The Tabular environment}
A simple group of cells:

\begin{tabular}{|c||c||||||||||||||||} %c, l, r denote colums, | denotes borders. You can double border... or put as many as you'd like.
\hline  1 & 2 \\ %\hline makes a horizontal line.
\hline 4 & 3 \\ %the & is the "next column" symbol.
5 & 6\\ %the \\ is the "next row" symbol
 & 8 \\ % If you want a blank cell, you still need the correct number of ampersands
\hline
\end{tabular}
As you can see, the markup is somewhat visual. But it is just an inline group of cells, not a real table!

\subsection{The Table environment}

Nesting a tabular environment in a table environment gives us what we want.

\begin{table}[h] %Begins a table. the "h" option means "try to place this Here, where I've typed it." "t" and "b" mean, place this at the top or the bottom of a page. This works for figures too.
	\centering %center the whole table
	\caption{Things I saw at the Grocery Store.} %table title
	\begin{tabular}{l c c} %3 columns, left-aligned, centered, and centered. No vertical seperators.
		\hline item & weight & cost\\
		\hline bananas & 1~lb & \$0.69\\
		lenovo computers & 3~lbs & \$400\\
		nachos & 0.5~lbs & \$700\\
		\hline
	\end{tabular}
	\label{tab:grocery} %remember to give the label a unique name!
\end{table}

Table \ref{tab:grocery} is an advanced table over the previous one (D. Voytenko, pers. comm.) Notice the items are left--aligned, while the other columns are centered.

\end{document}